\documentclass[margin,line]{res}

\usepackage{xcolor}

\usepackage[greek, english]{babel}
\usepackage[utf8x]{inputenc}
\usepackage[hidelinks]{hyperref}
%\usepackage{titlesec}

\newcommand{\g}{\textgreek}
\newcommand{\ver}{\vspace*{-2.7mm}}
\renewcommand{\baselinestretch}{1.0}
%\renewcommand{\baselineskip}{1.5in}

\oddsidemargin -0.7in
\evensidemargin -.7in
\textwidth=6.2in
\textheight=690pt
\voffset=-0.4in
\hoffset=1.4in
\itemsep=0in
\parsep=0in
% if using pdflatex:
\setlength{\pdfpagewidth}{\paperwidth}
\setlength{\pdfpageheight}{\paperheight}

\newenvironment{list1}{
  \begin{list}{\ding{113}}{%
      \setlength{\itemsep}{0in}
      \setlength{\parsep}{0in} \setlength{\parskip}{0in}
      \setlength{\topsep}{0in} \setlength{\partopsep}{0in}
      \setlength{\leftmargin}{0.17in}}}{\end{list}}
\newenvironment{list2}{
  \begin{list}{$\bullet$}{%
      \setlength{\itemsep}{0in}
      \setlength{\parsep}{0in} \setlength{\parskip}{0in}
      \setlength{\topsep}{0in} \setlength{\partopsep}{0in}
      \setlength{\leftmargin}{0.2in}}}{\end{list}}
\newenvironment{list3}{
  \begin{list}{}{%
      \setlength{\itemsep}{0in}
      \setlength{\parsep}{0in} \setlength{\parskip}{0in}
      \setlength{\topsep}{0in} \setlength{\partopsep}{0in}
      \setlength{\leftmargin}{0.2in}}}{\end{list}}
\definecolor{urlblue}{HTML}{0000EE}


\begin{document}

\name{ Nikolaos Ignatiadis - CV \vspace*{.1in}}

\begin{resume}
\section{\sc Contact Details}
\vspace{.05in}
\begin{tabular}{@{}p{2in}p{4in}}
Stanford University             & {Telephone:}  +1 (650) 656-0855 \\
Department of Statistics   & {E-mail:}    ignat@stanford.edu \\
390 Jane Stanford Way, &  {Website:} \href{https://nignatiadis.github.io/}{https://nignatiadis.github.io/}\\
Stanford, CA 94305, USA  & {Google Scholar:} \href{https://scholar.google.com/citations?user=KH3jpkoAAAAJ}{user=KH3jpkoAAAAJ} \\
\end{tabular}



\section{\sc Research Interests}
I am interested in the development of interpretable statistical methods, accompanied by robust software implementations, for the analysis of datasets generated from modern, high-throughput technologies. From a statistical perspective, this interest encompasses Empirical Bayes analysis, causal inference, multiple testing and statistics in the presence of contextual side-information.

\section{\sc Education}
{\bf Stanford University} \hfill Stanford, California, USA\\
%{\em Department of Statistics}
\vspace*{-.14in}
\begin{list2}
\item
\textbf{Ph.D. in Statistics.}  (GPA 4.24) \hfill  09/2016 -- present\\
Thesis advisor: Stefan Wager
\end{list2}

\vspace*{-2.5mm}
{\bf {Heidelberg University}} \hfill  {Heidelberg, Germany}\\
\vspace*{-.14in}
\begin{list2}
\item \textbf{M.Sc. Scientific Computing}, Grade 1.0 \hfill 2015 - 2016\\
Thesis advisors: Wolfgang Huber and Enno Mammen 
\item \textbf{B.Sc. Mathematics}, Grade 1.0 with \emph{distinction} \hfill 2011 - 2015
\item \textbf{B.Sc. Molecular Biotechnology}, Grade 1.0 \hfill 2010 - 2013
\end{list2}


\section{\sc Awards}
\textbf{Ric Weiland Graduate Fellowship  in the Humanities \& Sciences } \hfill 2020 - 2021\\
This fellowship is awarded to exceptional rising fourth year doctoral candidates in the humanities, social sciences, mathematics, and
statistics upon departmental or programmatic nomination.\\
\textbf{Departmental Teaching Assistant Award}, Statistics Department, Stanford  \hfill  2018\\
\textbf{iGEM Grand Prize Winner \& Best Foundational Advance} \hfill  2013 \\
The International Genetically Engineered Machine competition with Team Heidelberg at MIT.\\
\textbf{Deutschlandstipendium}, Heidelberg University, Stanford \hfill 2011 - 2013\\
This scholarship is awarded to talented and high-achieving students at public and state recognised institutions of higher education in Germany and is supported by the German Federal Government.






\section{\sc Journal Publications}
\begin{list1}
\item[1.] Ignatiadis, N., Saha, S., Sun D. L. and Muralidharan, O. (2021). \textbf{Empirical Bayes mean estimation with nonparametric errors via order statistic regression on replicated data.} Journal of the American Statistical Association (forthcoming).
\item[2.] Ignatiadis, N. and Huber, W. (2021). \textbf{Covariate powered cross-weighted multiple testing.} Journal of the Royal Statistical Society: Series B, 83, 720-751.
\item[3.] Karacosta, L. G., Anchang, B., Ignatiadis, N., \emph{et al.} (2019). \textbf{Mapping lung cancer epithelial-mesenchymal transition states and trajectories with single-cell resolution.} Nature Communications, 1010, 5887.
\item[4.]  Ignatiadis, N., Klaus, B., Zaugg, J. B. and Huber, W. (2016). \textbf{Data-driven hypothesis weighting increases detection power in genome-scale multiple testing.} Nature Methods, 13(7), 577-580.
\item[5.] Beer, R., Herbst, K., Ignatiadis, N., Kats, I., \emph{et al.} (2014). \textbf{Creating functional engineered variants of the single-module non-ribosomal peptide synthetase IndC by T domain exchange.} Molecular BioSystems, 10(7), 1709-1718.
\end{list1}
\section{\sc Conference proceedings}
\begin{list1}
\item[6.] Ignatiadis, N. and Wager, S. (2019). \textbf{Covariate-Powered Empirical Bayes Estimation.} Advances in Neural Information Processing Systems 32 (NeurIPS 2019)
\end{list1}
\section{\sc Preprints}
\begin{list1}
\item[7.] Ignatiadis, N. and Wager, S. (2021). \textbf{Confidence Intervals for Nonparametric Empirical Bayes Analysis.} \href{https://arxiv.org/abs/1902.02774}{arXiv:1902.02774}, Major revision submitted to Journal of the American Statistical Association.
\item[8.] Eckles, D., Ignatiadis, N., Wager, S. and Wu, H. (2021). \textbf{Noise-Induced Randomization in Regression Discontinuity Designs.} \href{https://arxiv.org/abs/2004.09458}{arXiv:2004.09458}, Submitted to Econometrica.
\item[9.] Ignatiadis, N. and Lolas, P.  (2021). \textbf{$\sigma$-Ridge: group-regularized ridge regression via empirical Bayes noise level cross-validation.} \href{https://arxiv.org/abs/2010.15817}{arXiv:2010.15817}.
\end{list1}


\section{\sc Invited discussions}
\textbf{International Seminar on Selective Inference (ISSI)} \hfill December 2020\\
Discussant of the talk `Clipper: p-value-free FDR control on high-throughput data from two conditions' 
by Prof. Jingyi Jessica Li.

\section{\sc Software}
{\bf R packages} in Bioconductor:
\begin{list2}
\item \href{https://bioconductor.org/packages/IHW}{{\color{urlblue} IHW}}:  Independent Hypothesis Weighting for multiple testing with side-information.
\item \href{https://bioconductor.org/packages/IHWpaper}{{\color{urlblue} IHWpaper}}: Companion to the IHW package facilitating reproducibility.
\end{list2}
{\bf Julia packages} in the official registry:
\begin{list2}
\item \href{https://github.com/nignatiadis/Aurora.jl}{{\color{urlblue} Aurora.jl}}: Empirical Bayes mean estimation with nonparametric errors on replicated data.
\item \href{https://github.com/nignatiadis/Empirikos.jl}{{\color{urlblue} Empirikos.jl}}: Nonparametric empirical Bayes confidence intervals.
\item \href{https://github.com/nignatiadis/RegressionDiscontinuity.jl}{{\color{urlblue}  RegressionDiscontinuity.jl}}: Basic functionality for analyzing sharp regression discontinuity designs. 
\item \href{https://github.com/nignatiadis/SigmaRidgeRegression.jl}{{\color{urlblue} SigmaRidgeRegression.jl}}: $\sigma$-Ridge for regression with features that can be partitioned into groups.
\item \href{https://github.com/nignatiadis/SmoothingSplines.jl}{{\color{urlblue} SmoothingSplines.jl}}: Nonparametric regression using smoothing splines.
\item Contributions to Distributions.jl, GLM.jl, Lasso.jl, MultipleTesting.jl and others.
\end{list2}


\section{\sc Industry experience}
\textbf{Google AdsMetrics}, Mountain View, USA \hfill Summer 2019\\
Data science intern with Omkar Muralidharan, Sujayam Saha and Dennis L. Sun. 


\section{\sc Research appointments}
{\bf Biomedical Informatics},  Stanford, California, USA \hfill 2021 - Present\\
Research assistant in the group of Prof. Nigam Shah funded by the NHLBI R01 grant `Applying statistical learning tools to personalize cardiovascular treatment'.\\
{\bf European Molecular Biology Laboratory},  Heidelberg, Germany \hfill 2014 - 2016\\
Research assistant in the group of Dr. Wolfgang Huber.

\section{\sc Talks and Presentations}


\textbf{Noise-Induced Randomization in Regression Discontinuity Designs.} \hfill August 2021\\
Joint Statistical Meetings (JSM): Causal Inference When Resources Are Limited\\
Virtual presentation\\
\textbf{$\sigma$-Ridge: group regularized ridge regression via empirical Bayes} \hfill April 2021\\
 \textbf{noise level cross-validation.}\\
Statistics seminar at Vrije Universiteit (VU) Amsterdam campus\\
Virtual presentation\\
\textbf{Confidence Intervals for Nonparametric Empirical Bayes Analysis.} \hfill April 2021\\
International Seminar on Selective Inference (ISSI)\\
Virtual presentation\\
\textbf{Bias-Aware Confidence Intervals for Empirical Bayes Analysis.} \hfill August 2020\\
Joint Statistical Meetings (JSM): Causality in Statistical Data Science\\
Virtual presentation\\
\textbf{Covariate-Powered Empirical Bayes Estimation.} \hfill January 2020\\
Blue seminar at the European Molecular Biology Laboratory\\
European Molecular Biology Laboratory (EMBL), Heidelberg, Germany\\
\textbf{Covariate-Powered Empirical Bayes Estimation.} \hfill December 2019\\
11th International Conference on Multiple Comparison Procedures\\
National Taiwan University (NTU), Taipei, Taiwan\\
\textbf{Bias-Aware Confidence Intervals for Empirical Bayes Estimation.} \hfill May 2019\\
Atlantic Causal Inference Conference (ACIC)\\
McGill University, Montreal, Canada\\
\textbf{Covariate powered cross-weighted multiple testing.} \hfill February 2019\\
Statistics Industrial Affiliates Conference\\
Stanford University, California, USA\\
\textbf{Covariate-powered cross-weighted multiple testing with FDR Control.} \hfill February 2018\\
Workshop: Post-selection Inference and Multiple Testing\\
Institut de Mathématiques de Toulouse, Toulouse, France\\
\textbf{MultipleTesting.jl: Simultaneous Statistical Inference in Julia.} \hfill June 2017\\
Lightning talk at JuliaCon\\
Berkeley, California, USA





\section{\sc Teaching}
\textbf{Instructor} at Stanford\\
STATS 302: Applied Statistics PhD Qualifying Exam Workshop. \hfill Summer 2020

\ver
\textbf{Teaching Assistant (TA)} at Stanford\\
STATS 315B: Modern Applied Statistics: Data Mining. \hfill Spring 2021\\ 
STATS 361: Causal Inference. \hfill Spring 2020\\
STATS 305B: Applied Statistics II. \hfill Winter 2020\\
STATS 315A: Modern Applied Statistics: Learning. \hfill Winter 2019\\
STATS 300A: Theory of Statistics I. \hfill Fall 2018\\
STATS 366 (BIOS 221): Modern Statistics for Modern Biology. \hfill Summer 2017 \& 2018, Fall 2019\\
STATS 218: Introduction to Stochastic Processes II. \hfill Spring 2018\\
STATS 290: Computing for Data Science. \hfill Winter 2018\\
STATS 305A: Introduction to Statistical Modeling. \hfill Fall 2017\\
STATS 191: Introduction to Applied Statistics. \hfill Winter 2017\\
STATS 141 (BIOS 141): Biostatistics.\hfill Fall 2016

\ver
\textbf{Trainer} at EMBL (European Molecular Biology Laboratory)\\
Introductory Course: Statistical Bioinformatics using R and Bioconductor \hfill October 2015




\section{\sc Professional Service}

\textbf{Journal peer review}\\
Annals of Statistics, Bernoulli, Bioinformatics, Biometrics, Biometrika, Electronic Journal of Statistics, Journal of the American Statistical Association, Operations Research, PeerJ, Statistical Science

\ver
\textbf{Conference peer review}\\
AISTATS 2021








\end{resume}
\end{document}
