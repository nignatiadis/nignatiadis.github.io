\documentclass[margin,line]{res}

\usepackage[greek, english]{babel}
\usepackage[utf8x]{inputenc}
\usepackage{hyperref}
%\usepackage{titlesec}

\newcommand{\g}{\textgreek}
\newcommand{\ver}{\vspace*{-2.7mm}}
\renewcommand{\baselinestretch}{1.0}
%\renewcommand{\baselineskip}{1.5in}

\oddsidemargin -0.7in
\evensidemargin -.7in
\textwidth=6.2in
\textheight=690pt
\voffset=-0.4in
\hoffset=1.4in
\itemsep=0in
\parsep=0in
% if using pdflatex:
\setlength{\pdfpagewidth}{\paperwidth}
\setlength{\pdfpageheight}{\paperheight}

\newenvironment{list1}{
  \begin{list}{\ding{113}}{%
      \setlength{\itemsep}{0in}
      \setlength{\parsep}{0in} \setlength{\parskip}{0in}
      \setlength{\topsep}{0in} \setlength{\partopsep}{0in}
      \setlength{\leftmargin}{0.17in}}}{\end{list}}
\newenvironment{list2}{
  \begin{list}{$\bullet$}{%
      \setlength{\itemsep}{0in}
      \setlength{\parsep}{0in} \setlength{\parskip}{0in}
      \setlength{\topsep}{0in} \setlength{\partopsep}{0in}
      \setlength{\leftmargin}{0.2in}}}{\end{list}}


\begin{document}

\name{ Nikolaos Ignatiadis - CV \vspace*{.1in}}

\begin{resume}
\section{\sc Contact Details}
\vspace{.05in}
\begin{tabular}{@{}p{2in}p{4in}}
Stanford University             & {\tt Telephone:}  +1 (650) 656-0855 \\
Department of Statistics   & {\tt E-mail:}    ignat@stanford.edu \\
390 Serra Mall &  {\tt Github:} \url{https://github.com/nignatiadis}\\
Stanford, CA, U.S.A.  & {\tt Google Scholar:} \href{https://scholar.google.com/citations?user=KH3jpkoAAAAJ}{\tt user=KH3jpkoAAAAJ} \\
\end{tabular}



\section{\sc Research Interests}
I am interested in the development of interpretable statistical methods, accompanied by robust software implementations, for the analysis of datasets generated from modern, high-throughput technologies. From a statistical perspective, this interest encompasses multiple testing and Empirical Bayes inference in the presence of contextual side-information.

\section{\sc Education}
{\bf Stanford University} \hfill Stanford, California, U.S.A.\\
%{\em Department of Statistics}
\vspace*{-.14in}
\begin{list1}
\item[]
\textbf{Ph.D. in Statistics.}  (GPA 4.2+) \hfill  09/2016 -- present\\
Successful completion of qualifying exams.\\
Thesis Advisor: Stefan Wager
%\g{Θέμα ερευνητικής εργασίας (υπό διαμόρφωση)}: Generalized Donaldson-Thomas Invariants
\end{list1}

\vspace*{-2.5mm}
{\bf {Heidelberg University}} \hfill  {Heidelberg, Germany}\\
%{\em Department of Mathematics and Statistics}
\vspace*{-.14in}
\begin{list2}
\item \textbf{M.Sc. Scientific Computing}, Grade 1.0 \hfill 2015 - 2016
%\g{Κατάταξη: 41ος μεταξύ περίπου 220 φοιτητών}.
\item \textbf{B.Sc. Mathematics}, Grade 1.0 with \emph{distinction} \hfill 2011 - 2015
\item \textbf{B.Sc. Molecular Biotechnology}, Grade 1.0 \hfill 2010 - 2013
%\g{Kατάταξη στο τέλος του τρίτου έτους: 11ος μεταξύ 216 φοιτητών.}
\end{list2}

\vspace*{-2.5mm}
{\bf {{The American College of Greece}}} \hfill  {Athens, Greece}\\
\vspace*{-.14in}
\begin{list1}
\item[]
\textbf{Lykio with Apolytirio Eniaiou Lykiou}  \hfill 2010\\
Valedictorian
\end{list1}


\section{\sc Preprints}
\begin{list1}

\item[1.]Ignatiadis, N. and Huber, W. (2018). \textbf{Covariate powered cross-weighted multiple testing. arXiv preprint} arXiv:1701.05179.

\end{list1}


\section{\sc Publications}
\begin{list1}
\item[2.]  Ignatiadis, N., Klaus, B., Zaugg, J. B. and Huber, W. (2016). \textbf{Data-driven hypothesis weighting increases detection power in genome-scale multiple testing. Nature methods}, 13(7), 577-580.
\item[3.] Beer, R., Herbst, K., Ignatiadis, N., Kats, I., \emph{et al.} (2014). \textbf{Creating functional engineered variants of the single-module non-ribosomal peptide synthetase IndC by T domain exchange. Molecular BioSystems}, 10(7), 1709-1718.
\end{list1}

\section{\sc Talks and Presentations}
\begin{list1}
\item[1.] \textbf{Workshop: Post-selection Inference and Multiple Testing} \hfill February 2018\\
Institut de Mathématiques de Toulouse, France\\
Invited talk -- Covariate-Powered Cross-Weighted Multiple Testing with FDR Control
\item[2.] \textbf{JuliaCon}, Berkeley  (\url{http://www.youtube.com/watch?v=R8NEfWZAVmw}
) \hfill June 2017\\
Lightning talk -- MultipleTesting.jl: Simultaneous Statistical Inference in Julia
\item[3.] \textbf{International Symposium on Synthetic Biology} \hfill December 2013 \\German Cancer Research Center, Heidelberg, Germany\\
Presentation about Team Heidelberg's iGEM project
\end{list1}

\section{\sc Ongoing\\ Projects}
\begin{list1}

\item[1.] \textbf{Bias-Aware Confidence Intervals for Empirical Bayes Estimation}  (with Stefan Wager)\\
We develop confidence intervals that provide asymptotic frequentist coverage of empirical Bayes estimands. Our intervals include an honest assessment of bias even in situations where empirical Bayes point estimates may converge very slowly.
\item[2.] \textbf{Covariate-powered Empirical Bayes shrinkage} (with Stefan Wager)\\
Empirical Bayes methods provide a practical way of improving point estimates by sharing information across units; say genes in a genomics experiment or multiple A/B tests. Here we develop practical methods for shrinkage estimation in situations with strong prior heterogeneity which can be explained using auxiliary covariates, such as the location of each gene or the type of each product being advertised.
\item[3.] \textbf{Estimation of sparse transition matrices} (with Sylvia Plevritis and Robert Tibshirani)\\
We develop methods using $L_1$ penalization for estimating transition matrices of discrete Markov models, when the number of states is large relative to the number of time points and observations. Furthermore, we extend our methods to deal with aggregate, incomplete data schemes and apply them to single cell data of the epithelial-mesenchymal transition, a key process which enables the metastasis of cancer cells.
\end{list1}

\section{\sc Teaching}

\textbf{Teaching Assistant (TA)} at Stanford\\
STATS 300A: Theory of Statistics I. \hfill Fall 2018\\
STATS 366 (BIOS 221): Modern Statistics for Modern Biology. \hfill Summer 2017 \& 2018\\
STATS 218: Introduction to Stochastic Processes II. \hfill Spring 2018\\
STATS 290: Computing for Data Science. \hfill Winter 2018\\
STATS 305A: Introduction to Statistical Modeling. \hfill Fall 2017\\
STATS 191: Introduction to Applied Statistics. \hfill Winter 2017\\
STATS 141 (BIOS 141): Biostatistics.\hfill Fall 2016

\ver
\textbf{Trainer}\\
Introductory Course: Statistical Bioinformatics using R and Bioconductor \hfill October 2015\\ EMBL (European Molecular Biology Laboratory), Heidelberg, Germany

\section{\sc Professional Service}

\textbf{Peer review}\\
Annals of Statistics, Bioinformatics, PeerJ (\url{https://publons.com/author/1470395})

\section{\sc Scholarships}


\textbf{Deutschlandstipendium} \hfill 2011-2013\\
A scholarship for talented and high-achieving students at public and state recognised institutions of higher education in Germany supported by the German Federal Government.


\section{\sc Awards and Honors}
\textbf{Departmental Teaching Assistant Award}, Statistics Department, Stanford \hfill June 2018

\textbf{Grand Prize Winner \& Best Foundational Advance} in the \textbf{iGEM} \hfill November 2013 \\
(international Genetically Engineered Machine) competition with Team Heidelberg, MIT.

\ver
\textbf{Bronze medal} in the International Biology Olympiad (\textbf{IBO}), Changwon, South Korea. \hfill July 2010

\ver
\textbf{Rank 3} in the 6th National Biology Competition, Greece.\hfill May 2010
\ver

\textbf{Rank 8} in the 8th European Competition of the Ancient Greek language. \hfill June 2009




\section{\sc Languages}
\textbf{English} (Fluent), \textbf{German} (Native), \textbf{Greek} (Native)


\section{\sc Programming Languages}
\textbf{R}, \textbf{Julia}, Python, C


\section{\sc Open-Source Software}
\textbf{IHW} (\url{http://bioconductor.org/packages/IHW})\\ A R/Bioconductor package implementing the Independent Hypothesis Weighting method. \\
\textbf{IHWpaper} (\url{http://bioconductor.org/packages/devel/data/experiment/html/IHWpaper.html})\\
A package reproducing all analyses for the Independent Hypothesis Weighting publications.\\
\textbf{SmoothingSplines.jl} (\url{https://github.com/nignatiadis/SmoothingSplines.jl})\\
A statistical package for nonparametric regression via Smoothing Splines in Julia.

\end{resume}
\end{document}
